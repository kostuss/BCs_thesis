\newpage % Rozdziały zaczynamy od nowej strony.
\section{Wstęp}
\par Ciągły rozwój nauki i techniki sprawia, że wiele dziedzin życia zmienia się na naszych oczach. Obserwujemy coraz to większe dążenie do automatyzacji i informatyzacji otaczającego nas świata. Zmiany, które obserwujemy nie dotyczą jedynie przemysłu czy wąskiego grona wysoko-wyspecjalizowanych profesji ale często wkraczają już nawet do naszego życia codziennego. Komputery i roboty są  w stanie przeprowadzić za nas skomplikowaną reakcję chemiczną ale także wymieszać składniki na ciasto w odpowiednich proporcjach. 

\par Wspomniane proporcje są tutaj kluczowe stąd też z automatyzacją nieodzownie łączą się takie zagadnienia jak sterowanie i regulacja. Praktycznie każde mniej lub bardziej skomplikowane zadanie możemy zapisać w postaci konkretnego algorytmu, a następnie zrealizować go poprzez kontrolę pewnych ustalonych parametrów. Doskonałym przykładem jest tutaj dostosowanie temperatury wody w wannie lub prowadzenie samochodu. W obu tych przypadkach dążymy do osiągnięcia pewnego pożądanego rezultatu, prawidłowej temperatury wody lub odpowiedniego toru jazdy.

\par Automatyka rozwija się już od średniowiecza i w tym czasie można wyszczególnić wiele przełomowych momentów. Za jedne z głównych możemy uznać: prace J. C. Maxwella z zakresu stabilności regulacji (1863 r.), zapoczątkowanie metod częstotliwościowych analizy i syntezy układów przez H. Nyquista (1932 r.) czy w końcu zastosowanie regulacji w strukturze zamkniętej ze sprzężeniem zwrotnym poprzez opracowanie regulatora PID. Obecnie jedną z powszechnie uznanych i stosowanych metod jest regulacja predykcyjna. Jednak potrzeba automatyzacji coraz to nowych dziedzin naszego życia wymusza ciągły rozwój w obszarze automatyki i konieczność udoskonalania istniejących rozwiązań lub tworzenia zupełnie nowych. Stosowane algorytmy mogą okazać się niewystarczające gdy zadania, które przed nimi stawiamy staną się bardziej złożone i wymagać będą uwzględnienia wielu niezależnych czynników. Z pomocą mogą przyjść tutaj niewątpliwie liczne osiągnięcia w dziedzinie przetwarzania informacji, a za jedno z największych należy uznać sztuczne sieci neuronowe.

\par Rosnąca popularność i uznanie sztucznych sieci neuronowych wiąże się z ich zdolnością łatwego adaptowania się do rozwiązywania różnorodnych problemów obliczeniowych. Możliwość odwzorowania nauczonych wzorców i generalizacji nabytej wiedzy sprawia, że stały się szczytowym osiągnięciem w obszarze sztucznej inteligencji. Z tego względu oczywistym wydaje się chęć wykorzystania ich zalet i sprawdzenia zdolności adaptacyjnych także w obszarze regulacji.
  
\par Głównym celem pracy jest zastosowanie jednej z metod sztucznej inteligencji w obszarze regulacji predykcyjnej. Dzięki zastosowaniu możliwie prostych i podstawowych struktur, a za taką możemy uznać jednokierunkową sztuczną sieć neuronową, możliwe będzie przedstawienie ogólnego potencjału wykorzystania sztucznych sieci neuronowych do różnorodnych zadań stawianych przed regulatorami. Ważnym aspektem będzie także wykorzystanie algorytmu upraszczania sztucznej sieci neuronowej, który ma za zadanie poprawić możliwość generalizacji danego problemu przez sztuczną sieć neuronową. 

\par W pierwszym rozdziale pracy przybliżona zostanie literatura przedmiotu, przegląd aktualnych rozwiązań i omówienie możliwości ich uzupełnienia. Następnie zaprezentowane zostaną wszelkie założenia teoretyczne, szczegółowy opis stosowanych struktur i sposób ich wykorzystania. W kolejnym rozdziale znajdą się właściwe wyniki przeprowadzonych eksperymentów. Na końcu uzyskane rezultaty zostaną podsumowane co pozwoli na wyciągnięcie ogólnych wniosków z niniejszej pracy. 



