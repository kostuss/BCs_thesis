\newpage % Rozdziały zaczynamy od nowej strony.
\section{Podsumowanie}
\par Praca w swoim założeniu miała na celu dokonanie możliwie uniwersalnej weryfikacji zdolności sieci neuronowej do generalizacji zadania regulacji. Jak zostało wskazane w przeglądzie literatury istnieją przykłady udanego wykorzystania sztucznych sieci neuronowej w obszarze automatyki jakim jest sterowanie. Dostępne prace skupiają się na opisaniu praktycznych zastosowań różnych rodzajów sieci neuronowych do regulacji wysoce dynamicznych procesów. Niniejsza praca natomiast wpisuje się w pewną niszę polegająca na braku prac przedstawiających możliwie ogólny charakter porównania. 
\par Kierując się wskazanymi założeniami, w pracy zdecydowano się na dokonanie porównania pomiędzy dwoma możliwie prostymi algorytmami z dziedziny sztucznych sieci neuronowych oraz regulacji predykcyjnej. Za dobrego reprezentanta pierwszej grupy wybrano jednokierunkową sztuczną sieć neuronową z jedną warstwą ukrytą. Natomiast oczywistym wyborem w przypadku tradycyjnego podejścia do zagadnienia regulacji predykcyjnej była jedna z najczęściej stosowanych metod tego typu czyli algorytm DMC. Dodatkowo mając na celu otrzymanie możliwie uniwersalnego aproksymatora zadania regulacji zdecydowano się wykorzystać w pracy metodę redukcji sieci neuronowej. Przycinanie wag w swoim założeniu ma na celu zwiększyć zdolność generalizacji danego zagadnienia poprzez wyeliminowanie niepotrzebnych połączeń międzyneuronowych. W tej dziedzinie powszechne uznanie zyskał algorytm OBD i to właśnie on został wykorzystany w trakcie badań. 
\par Niewątpliwą wartością dodaną niniejszej pracy jest samodzielna implementacja wykorzystywanych struktur. Pozwoliło to na pełniejsze zrozumienie tematyki badania ale za to niewątpliwie wiąże się z utratą wydajności w stosunku do metod implementowanych w zewnętrznych bibliotekach. Biorąc jednak pod uwagę podstawowe zadanie regulacji, które zostało wykorzystane do porównania dwóch metod, nie stanowi to istotnego ograniczenia i nie powinno być uznane za wadę. Zaimplementowane struktury pozwoliły w pierwszej kolejności na wygenerowanie i odpowiednie przeskalowanie danych uczących i weryfikujących. Warto podkreślić, że oba zbiory generowane były w sposób niezależny co pozwala na weryfikację działania sieci pod kątem ogólnego zadania regulacji, a nie jedynie odwzorowania konkretnych przebiegów.
\par Dalsze prace podzielone zostały na trzy etapy. W pierwszej części skupiono się na doborze odpowiedniej architektury sieci pozwalającej na optymalną generalizację zadania regulacji. Efektem tych prac był wybór struktury opartej o 150 neuronów warstwy ukrytej, a za zmienne na podstawie których odbywa się sterowanie przyjęto 30 wartości uchybu regulacji oraz aktualną wartość zadaną. Następnie zbadano zależność funkcji kosztu od liczby iteracji algorytmu uczenia sieci, pozwoliło to na zdefiniowanie spójnego kryterium zakończenia uczenia, które skutkowało kończeniem procedury po około 200 epokach. Warto zauważyć, że kryterium wyjścia dotyczy danych uczących, a nie weryfikujących jak w przypadku większości rozwiązań z dziedziny uczenia maszynowego. Modyfikacja wynika wprost z założenia konieczności osiągnięciu minimum funkcji kosztu przed rozpoczęciem procedury OBD. Po zakończonym etapie uczenia dokonano wstępnego porównania wyselekcjonowanej struktury sieci z regulatorem DMC. Wyniki wskazują na zbliżoną efektywność obu metod, z pewnymi niedoskonałościami działania sieci neuronowej w obszarze małych wartości zadanych.
\par Drugi etap prac dotyczył zastosowania procedury redukcji sieci neuronowej. W trakcie eksperymentów wykazano, że redukcja blisko 55\% połączeń międzyneyuronowych przyczynia się do około 90\% procentowej redukcji funkcji kosztu na danych weryfikujących. Niewątpliwie jest to jednoznaczne udowodnienie korzyści wynikających z zastosowania algorytmu OBD. Po skutecznym uproszczeniu sieci dokonano pełnego porównania dwóch podejść do zagadnienia regulacji. Wyniki pokazują, że sieć neuronową w pełni poradziła sobie z zadaniem regulacji, a co ważne udało się wyeliminować początkowe problemy występujące w obszarze małych wartości zadanych. Za jedyną niedoskonałość regulacji opartej o sieć neuronową należy uznać problem z długotrwałym utrzymaniem wartości sterowania na stałej wartości. Analizując wartości błędów MSE należy przyjąć, że problem ma charakter marginalny nie mniej jednak wskazuje to na zasadność wykorzystywania regulacji opartej o sieci neuronowe głównie w przypadku wysoce dynamicznych procesów, co w ogólności znajduje swoje potwierdzenie w dokonanym przeglądzie literatury.
\par Zakończeniem prac było zweryfikowanie odporności dwóch porównywanych regulatorów na zmiany obiektu regulacji. W pracy za przykład obiektu posłużył człon inercyjny drugiego rzędu z opóźnieniem. Wykazano, że zmiana opóźnienia w znaczący sposób przyczynia się do wzrostu błędu regulacji dla obu metod. Jednak należy stwierdzić, że w obrobię małych modyfikacji sieć neuronowa poradziła sobie wyraźnie lepiej od algorytmu DMC i należy uznać to za jedną z głównych przewag badanej struktury. Przyczynę takiej własności można wiązać z wykorzystaną procedurą upraszczania sieci, jest to jednak jedynie hipoteza, która wymaga weryfikacji w czasie przyszłych prac nad zagadnieniem. Zmiany stałych czasowych obiektów regulacji wiązały się z mniejszym zakłóceniami działania algorytmów i należy uznać, że obie metody poradziły sobie w tym przypadku porównywalnie dobrze.
\par W ogólności niniejsza praca udowadnia, że zdolność sztucznych sieci neuronowych do generalizacji różnorodnych zagadnień obliczeniowych może z powodzeniem być wykorzystywana w obszarze regulacji. Niewątpliwe ważnym aspektem było wskazanie na korzyści wynikające z zastosowania procedury upraszczania połączeń międzyneuronowych. Praca miała jednak charakter ogólnego i uniwersalnego porównania, z tego względu należałoby w ramach przyszłej pracy skupić się nad uwzględnieniem w analizie wysoce dynamicznych procesów oraz dokładniej zbadać wpływ algorytmu OBD na redukcję wrażliwości sieci pod kątem modyfikacji układów regulacji. 