\newpage % Rozdziały zaczynamy od nowej strony.
\section{Przegląd literatury przedmiotu}
Obszerność dostępnych opracowań z zakresu automatyki jak i sztucznych sieci neuronowych sprawia, że omówienie istniejących rozwiązań zostanie podzielone na dwa etapy. Pierwszym z nich jest przegląd prac z zakresu sztucznych sieci neuronowych. Omówione zostaną zarówno pozycje teoretyczne wprowadzające w dziedzinę sztucznej inteligencji i pozwalające poczynić niezbędne założenia jak i przykłady praktycznego zastosowania wybranych struktur ze szczególnym uwzględnieniem tych, w których wykorzystane zostały algorytmy upraszczania sieci sieci neuronowych. Następnie przeanalizuję dostępną literaturę z zakresu regulacji predykcyjnej i wskaże przykłady zastosowania sieci neuronowych w tej dziedzinie.

\par Rozpoczynając pracę nad zagadnieniem sztucznych sieci neuronowych warto zapoznać się z dwoma pozycjami \cite{haykin1999} oraz \cite{osowski2013}. Oba opracowania stanowią kompleksowy przegląd istniejących metod i mogą stanowić punkt wyjścia do każdej analizy zajmującej się tematyką sztucznych sieci neuronowych. Lektura dwóch pozycji pozwoliła mi zdecydować się na wybór jednokierunkowej sieci typu sigmoidalnego z jedną warstwą ukrytą. Jest to możliwie prosta struktura, która pozwoli zarówno na samodzielną implementację jak i będzie stanowić dobry punkt wyjścia do możliwiej dalszej analizy z wykorzystaniem bardziej skomplikowanych struktur. Moją decyzję opieram także na \cite{hornik1991} gdzie autor wykazuje, że podstawowa struktura jaką jest jednokierunkowa sieć z jedną warstwą ukrytą może stanowić uniwersalny aproksymator przy założeniu dostatecznej liczby neuronów w warstwie ukrytej oraz prawidłowego doboru funkcji aktywacji. Szczegółowe umówienie wybranej struktury przedstawione zostanie w kolejnym rozdziale pracy. 

\par Kolejnym zagadnieniem jest wybór metody redukcji sieci i jak wskazuje \cite{osowskiOBD} za jedno z lepszych rozwiązań możemy uznać metody wrażliwościowe, a w szczególności metodę Optimal Brain Damage (OBD) zaproponowaną w pracy \cite{lecun1989}. Implementacja i sprawdzenie wskazanej metody w kontekście regulacji predykcyjnej będzie ciekawym zagadnieniem i jednym z głównych celów niniejszej pracy. Możemy wskazać przykłady prac, które pozwalają nam przypuszczać, że zastosowanie OBD rzeczywiście poprawi działanie proponowanej architektury i zwiększy zdolność generalizacji. W pracy \cite{chaber2018} autorzy badają efektywność zastosowania metody OBD do redukcji rekurencyjnej sieci neuronowej z jedną warstwą ukrytą. Dzięki zastosowaniu opisywanej techniki dokumentują redukcję 60 procent wag co przyczynia się do 30-to procentowego spadku błędu popełnianego przez model. Dodatkowo warto zwrócić uwagę, że analiza została przeprowadzona na podstawie modelowania chemicznej reakcji neutralizacji, a zatem na przykładzie procesu wysoce dynamicznego. Niniejsza praca zajmować się będzie ogólnym porównaniem dwóch metod regulacji z wykorzystaniem teoretycznego obiektu, jednak mimo to osiągnięte przez autorów rezultaty możemy z pewnym przybliżeniem traktować jako punkt odniesienia w trakcie dalszej analizy.

\par Następnie warto wskazać prace porównujące różne metody redukcji sieci neuronowych, a zwłaszcza na porównanie OBD z inna metodą wrażliwością Optimal Brain Surgeon (OBS) szczegółowo opisaną w \cite{osowskiOBD}. W pracy \cite{kavzoglu1998} porównana została efektywność trzech różnych metod redukcji sieci: Magnitude Based Pruning (MP), OBD i OBS. Analizę przeprowadzono z wykorzystaniem jednokierunkowej sieci neuronowe użytej do klasyfikacji obiektów z zdjęć terenu. Przedmiot analizy jest co prawda zupełnie odmienny jednak zbliżona architektura sieci stanowi tutaj podstawę do dokładniejszego przyjrzenia się wynikom uzyskanym przez autorów. Podobnie jak w przypadku poprzedniej pracy strukturę sieci udało zredukować się o 60 procent jednak tym razem wiązało się to jedynie z niewielką choć zauważalną poprawą klasyfikacji. Wartym zauważenia natomiast jest fakt, że OBS pozwolił na osiągnięcie najlepszych rezultatów spośród trzech metod. Co więcej nie jest to jedyna praca, w której wykazana zostaje taka zależność. Autorzy w pracy \cite{hassibi1993} porównując dwie wspomniane wcześniej metody redukcji sieci wskazują na istotny problem, a mianowicie, że OBD nie zawsze redukuje prawidłowe wagi sieci, a także w większości przypadków prowadzi do mniejszej redukcji sieci niż OBS. Analiza przeprowadzona została na podstawie przykładowego zbioru danych używanego do weryfikacji różnorodnych zagadnień uczenia maszynowego MONK's Problems Data Set. Na podstawie przedstawionych opracowań warto
rozważyć użycie nie tylko metody OBD ale poszerzenie analizy o OBS, jednak decyzja o dalszym rozwoju pracy uwarunkowana będzie jakością wyników uzyskanych za pomocą algorytmu OBD.

\par Po zapoznaniu się z pracami kluczowymi ze względu na aspekt doboru architektury sieci należy spojrzeć na przykłady zastosowania sieci neuronowych w regulacji predykcyjnej. W pracy \cite{kiti2009} autorzy badają wpływ wykorzystanie wielowarstwowej jednokierunkowej sieci neuronowej do kontroli nieliniowego wieloczynnikowego procesu chemicznego jakim jest wytrawianie metalu (Steel Pickling). Wykazane zostaje, że w przypadku procesu o podanej charakterystyce tradycyjne metody sterowania okazują się dawać gorsze rezultaty niż podejście oparte o sztuczne sieci neuronowe, które to przykładowo zdecydowanie lepiej radziły sobie z eliminacją oscylacji. 
\par Kolejnym przykładem zastosowania sieci neuronowych do regulacji procesu chemicznego jest praca \cite{hosen2011} W tym przypadku regulacja ponownie dotyczy złożonego nieliniowego procesu chemicznego, a w celu jego regulacji użyto rekurencyjnej sieci neuronowej z jedną warstwą ukrytą. Warto zauważyć tutaj fakt, że regulacja oparta o sieci neuronowe porównana zostaje z klasycznym regulatorem PID. Klasyczne podejście prowadzi do wyraźnego przeregulowania  układu oraz zmniejsza stabilność układu. W artykule jednoznacznie wykazana zostaje wyższość metody regulacji opartej o sztuczne sieci
neuronowe jednak autorzy wskazują, że jest to głównie związane z dużą złożonością regulowanego procesu. Na tej podstawie możemy przypuszczać, że obserwować będziemy coraz więcej problemów regulacji gdzie tradycyjne metody mogą okazać się niewystarczające i szukanie rozwiązań opartych o sieci neuronowe stanie się niezbędne. 
\par Ostatnim z przykładów, który chce omówić jest praca \cite{afram2017}, w której zaprezentowano przykład wykorzystania regulacji opartej o sieci neuronowe do sterowania system inżynierii sanitarnej w budynkach mieszkalnych. Autorzy analizują wyniki uzyskane za pomocą kilku różnych architektur sieci, a jako najbardziej typowy wybór podają jednokierunkową sieć o jednej lub więcej warstwach ukrytych. Przeprowadzona analiza wskazuje, że zastosowanie regulatora opartego o sztuczne sieci neuronowe pozwoliło zredukować konsumpcję energii w zakresie nawet do 50\% w zależności od analizowanego przypadku.

\par Wyniki wskazanych powyzej prac \cite{kiti2009}, \cite{hosen2011} oraz \cite{afram2017} jednoznacznie wskazują na korzyści wynikające z zastosowania sztucznych sieci neuronowych w systemach regulacji. Architektura sieci w przypadku żadnej z prac nie odpowiadała tej wybranej do mojej pracy, co więcej w żadnym z podejść nie został wykorzystany algorytm OBD lub OBS. Pokazuje to, że przeprowadzenie analizy w takim zakresie może nieść za sobą duża wartość zarówno naukową jak i poznawczą. 


