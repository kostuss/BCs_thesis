\newpage % Rozdziały zaczynamy od nowej strony.
\section{Wyniki pracy}
Niniejszy rozdział prezentuje główne wyniki uzyskane w trakcie badania i zawiera w sobie podsumowanie przeprowadzonych eksperymentów. Omówione zostaną w nim wszystkie uzyskane rezultaty, które pozwalają na ocenę czy sztuczne sieci neuronowe mogą być stosowane z powodzeniem w obszarze regulacji. Na początku zaprezentowany zostanie sposób generacji danych, poczynione założenia i sposób działania regulacji opartej o sieć neuronową. Kolejno wybrana zostanie optymalna liczba neuronów warstwy ukrytej pozwalająca na minimalizację funkcji celu z zachowaniem ogólności rozwiązania. W kolejnej części zbadany zostanie wpływ zastosowania redukcji sieci na osiągane przez nią rezultaty. Po wybraniu optymalnej struktury i pełnym wytrenowaniu sieci zbadamy jak poradzi sobie z regulacją obiektów, które nie znalazły odzwierciedlenia w przykładach uczących. Ciekawym eksperymentem będzie też sprawdzenie działania sieci w przypadku wielokrotnych skoków wartości zadanej. Pod koniec rozdziału sformułowane zostaną ogólne wnioski i uwagi płynące z całości eksperymentów, które znajdą również swoje odzwierciedlenie w późniejszym podsumowaniu pracy.

\subsection{Generacja danych}

\subsection{Wybór struktury sieci}

\subsection{Zastosowanie algorytmu OBD}

\subsection{Zastosowanie innych obiektów regulacji}
